\documentclass[11pt]{article} % Try also "scrartcl" or "paper"
\usepackage{helvet}
\linespread{1}
 \usepackage[margin=1.2cm]{geometry}   % to change margins
 \usepackage{titling}             % Uncomment both to   
 \setlength{\droptitle}{-2.3cm}     % change title position 
\usepackage[affil-it]{authblk}
\usepackage{bibentry}
\usepackage{cite}
\usepackage{graphicx,bm,times,subfig,amsmath,amsfonts,listings,url}
\usepackage{color}
\usepackage[page]{appendix}

\definecolor{mygreen}{rgb}{0,0.6,0}
\definecolor{mygray}{rgb}{0.5,0.5,0.5}
\definecolor{mymauve}{rgb}{0.58,0,0.82}
\author{Ben Crabbe}
\affil{University of Bristol, UK}
\title{%\vspace{-1.5cm}            % Another way to do
Java Database Report}
\begin{document}


\maketitle
\section{Records}
This covers the first step of the assignment, designing a Record class.

I decided that, to be true to the relational model, the order of the fields in each record should be unimportant, therefore I stored them in a Set but I choose the LinkedHashSet so that when a certain order seemed logical the fields could be defined in that order and would be printed etc in that order automatically.

I also thought that it seemed the record should know what its fields were and not just their attributes, so I created the Attribute class, and since all instances of this class must belong to a Record I made it an inner class. 

I also created a custom exception to throw when any of the Record methods was used incorrectly, such as adding a field which already exists, or setting the value of a field which doesn't exist etc. 



\lstset{language=Java}
\lstdefinestyle{customc}{
  belowcaptionskip=1\baselineskip,
  breaklines=true,
  frame=L,
  xleftmargin=\parindent,
  language=Java,
  showstringspaces=false,
  basicstyle=\footnotesize\ttfamily,
  keywordstyle=\bfseries\color{mymauve},
  commentstyle=\itshape\color{purple!40!black},
  identifierstyle=\color{blue},
  stringstyle=\color{orange},
}
\lstset{ %
  backgroundcolor=\color{white},   % choose the background color; you must add \usepackage{color} or \usepackage{xcolor}
  basicstyle=\footnotesize,        % the size of the fonts that are used for the code
  breakatwhitespace=false,         % sets if automatic breaks should only happen at whitespace
  breaklines=true,                 % sets automatic line breaking
  captionpos=b,                    % sets the caption-position to bottom
  commentstyle=\color{mygreen},    % comment style
  extendedchars=true,              % lets you use non-ASCII characters; for 8-bits encodings only, does not work with UTF-8
  frame=single,                    % adds a frame around the code
  keepspaces=true,                 % keeps spaces in text, useful for keeping indentation of code (possibly needs columns=flexible)
  language=Java,                 % the language of the code
  numbers=left,                    % where to put the line-numbers; possible values are (none, left, right)
  numbersep=5pt,                   % how far the line-numbers are from the code
  numberstyle=\tiny\color{mygray}, % the style that is used for the line-numbers
  rulecolor=\color{black},         % if not set, the frame-color may be changed on line-breaks within not-black text (e.g. comments (green here))
  showspaces=false,                % show spaces everywhere adding particular underscores; it overrides 'showstringspaces'
  showstringspaces=false,          % underline spaces within strings only
  showtabs=false,                  % show tabs within strings adding particular underscores
  stepnumber=2,                    % the step between two line-numbers. If it's 1, each line will be numbered
  %stringstyle=\color{mymauve},     % string literal style
  tabsize=2,                       % sets default tabsize to 2 spaces
  title=\lstname                   % show the filename of files included with \lstinputlisting; also try caption instead of title
}

\begin{lstlisting}

import java.util.*;

class Record
{
   public class RecordException extends Exception
   {
      static final long serialVersionUID = 42L;
      public RecordException()
      {
         super();
      }

      public RecordException(String message)
      {
         super(message);
      }

      public RecordException(String message, Throwable cause)
      {
        super(message, cause);
      }

      public RecordException(Throwable cause)
      {
         super(cause);
      }
   }

   class Attribute
   {
      String name;
      String value;

      Attribute(String nameInput, String valueInput)
      {
         name = nameInput;
         value = valueInput;
      }
   }

    Set<Attribute> data;

    Record() {
      data = new LinkedHashSet<Attribute>();
    }

   void addField(String fieldName, String fieldValue) throws RecordException
   {
      for(Attribute field: data) {
         if(field.name==fieldName) {
            throw new RecordException("There already exists a field named " + fieldName + "in record.");
         }
      }
      Attribute newField = new Attribute(fieldName,fieldValue);
      this.data.add(newField);
   }

   String getFieldValue(String fieldName) throws RecordException
   {
      for(Attribute field: data) {
         if(field.name==fieldName) {
            return field.value;
         }
      }
      throw new RecordException("No field named " + fieldName + " in record.");
   }

   void setFieldValue(String fieldName, String fieldValue) throws RecordException
   {
      for(Attribute field: data) {
         if(field.name==fieldName) {
            field.value = fieldValue;
            return;
         }
      }
      throw new RecordException("No field named " + fieldName+ " in record.");
   }

   int countFields()
   {
      return data.size();
   }

   public static void main(String[] args)
   {
      Record row = new Record();
      row.testRecord();
   }

   void testRecord()
   {
      try {
         this.addField("Field1","value1");
         this.addField("Field2","value2");
         this.setFieldValue("Field2","value2.2");
         System.out.println(this.getFieldValue("Field1"));
         System.out.println(this.getFieldValue("Field2"));
         System.out.println( this.countFields() );
      } catch (Exception e) {
         System.out.println(e.getMessage());
      }
   }
}

\end{lstlisting}

\section{Tables}



\end{document}